\documentclass[20pt,a4paper]{article}
\usepackage[a4paper, top=2.5cm, bottom=2.5cm, left=2.5cm, right=2.5cm]{geometry}
\usepackage{amsmath,esint,mathtools}
\usepackage{amsfonts,amssymb}
\usepackage{listings}
%------------ ปรับสีลิงค์และ url -----------%/a
\usepackage{xcolor}

\usepackage[unicode=true]{hyperref}
\hypersetup{
    colorlinks,
    linkcolor={red!50!black},
    citecolor={blue!50!black},
    urlcolor={blue!80!black},
}
\renewcommand\UrlFont{\normalfont}
%------------ สำหรับการใช้ฟ้อนท์ภาษาไทย -----------%
\usepackage[no-math]{fontspec}
\usepackage{xunicode}
\usepackage{xltxtra}
\XeTeXlinebreaklocale "th"
\renewcommand{\baselinestretch}{1.6}
\setmainfont[Script=Thai,Scale=1.4,ItalicFont={THSarabunNew_Italic.ttf}, BoldFont={THSarabunNew_Bold.ttf},BoldItalicFont={THSarabunNew_BoldItalic.ttf} ]{THSarabunNew.ttf}
\setmonofont[Script=Thai,Scale=0.9]{DroidSansMono.ttf}
%--------------- เปลี่ยนชื่อรูปและตาราง -------------%
\usepackage[labelsep=space]{caption}
\renewcommand{\figurename}{รูปที่}  
\renewcommand{\tablename}{ตารางที่} 
%--------------- กำหนดรูปแบบการอ้างอิง -------------%
\usepackage[numbers,sort&compress]{natbib}
%-----------------------------------------------%
\author{พงษ์เทวิน นาคพงศ์พิมาน\\ภาควิชาคณิตศาสตร์และวิทยาการคอมพิวเตอร์ \\ คณะวิทยาศาสตร์ จุฬาลงกรณ์มหาวิทยาลัย}
\title{หลักการใช้ขั้นตอนวิธีตะแกรงกำลังสอง \\ ในการแยกตัวประกอบของจำนวนเต็ม}
\date{29 พฤศจิกายน 2564}
%-----------------------------------------------%
%                     START                     %
%-----------------------------------------------%
\begin{document}
\maketitle
%--------------------------------------
\section{บทนำ}
นักคณิตศาสตร์ได้ทำการทดลองคิดค้นวิธีใหม่ ๆ ในการแยกตัวประกอบจำนวนเต็มที่มีประสิทธิภาพมากขึ้นและเร็วขึ้นตั้งแต่สมัยโบราณกาล แรกเริ่มเดิมทีนักคณิตศาสตร์ทำได้แค่เพียงการนำจำนวนเฉพาะทดลองหารไปเรื่อย ๆ ซึ่งเป็นวิธีที่ขาดประสิทธิภาพในการแยกตัวประกอบจำนวนที่มีค่ามาก วิธีดังกล่าวถูกใช้เรื่อยมาจนกระทั่ง Pierre de Fermat นักคณิตศาสตร์ชาวฝรั่งเศส ได้คิดค้นวิธีแยกตัวประกอบจากสูตรของผลต่างกำลังสอง คือ $a^2-b^2 =(a+b)(a-b)$ ถ้าหากมี
$x\in \mathbb{Z}$ โดย $x^2 > N$ และ $x^2-N$ เป็นจำนวนกำลังสองแล้ว กำหนด $a^2 = x^2$ และ $b^2 = x^2 - N$ จะทำให้
\begin{equation}
    N = a^2 - b^2 = (a-b)(a+b) = \big(x-\sqrt{x^2-N}\big)\big(x+\sqrt{x^2+N}\big)
\end{equation}
ทำให้ $x\pm\sqrt{x^2-N}$ เป็นตัวประกอบของ $N$
\vspace{4mm}

แม้การคำนวนดังกล่าว เมื่อนำมาใช้งานตรง ๆ จะไม่ได้รวดเร็วขึ้นหรือแตกต่างจากการทดลองหารด้วยจำนวนเฉพาะมากนัก แต่ขั้นตอนวิธีนี้ได้เป็นพื้นฐานให้กับขั้นตอนวิธีสมัยใหม่อย่างมากมาย เช่น Continued Fraction Method, Quadratic Sieve และ Number Field Sieve ที่เป็นขั้นตอนวิธีที่ใช้กันอย่างแพร่หลายในวงการรหัสลับในปัจจุบัน ในบทความนี้จะกล่าวถึงการใช้ขั้นตอนวิธีตะแกรงกำลังสอง หรือ Quadratic Sieve และการทดลองนำไปใช้จริง
\section{ตะแกรงกำลังสอง}
ขั้นตอนวิธีตะแกรงกำลังสอง (Quadratic Sieve Algorithm หรือต่อจากนี้จะเรียกย่อว่า QS) ถูกค้นพบขึ้นในปี 1981 โดย Carl Pomerance \cite{pomerance} 
ได้นำหลักการแยกตัวประกอบที่ Maurice Kraitchik และ John D. Dixon ดัดแปลงมาจากการแยกตัวประกอบของ Fermat ขั้นตอนวิธีที่เกิดขึ้นถือว่าเป็นวิธีที่เร็วที่สุดวิธีหนึ่งก่อนการค้นพบ Number Field Sieve ในปี 1993 แต่อย่างไรก็ตาม QS ก็ยังเป็นขั้นตอนวิธีที่เร็วกว่า Number Field Sieve ในจำนวนที่มีจำนวนหลักไม่ถึง 110 หลักอยู่ดี
\section{หลักการของตะแกรงกำลังสอง}
นิยาม $n$ เป็นจำนวนเต็มที่ต้องการแยกตัวประกอบ ขั้นตอนวิธีนี้จะทำการหาจำนวนเต็ม $x$ และ $y$ ที่ $x\not\equiv \pm y\pmod n$ และ $x^2\equiv y^2\pmod n$ ค่า $x$ และ $y$ ดังกล่าวจะทำให้ $(x-y)(x+y)\equiv0\pmod n$ และจะมีโอกาสอย่างน้อย $\frac{1}{2}$ ที่จะพบค่า $(x-y,n)\neq 1$ เป็นตัวประกอบของ $n$ ซึ่งในการหาค่าดังกล่าว เราจะนิยาม
\begin{equation}
    Q(x) = (x+\lfloor\sqrt{n}\rfloor)^2 - n = \Tilde{x}^2 - n 
\end{equation}
แล้วทำการหาค่า $Q(x_1),Q(x_2),\ldots,Q(x_k)$ เพื่อเลือกเซตย่อย $\{i_1,i_2,\ldots,i_r\}$ ที่มีค่า $Q(x_{i_1})Q(x_{i_2})\ldots Q(x_{i_r}) = a^2$ เป็นจำนวนกำลังสอง ถ้าสังเกตจะเห็นว่า $Q(x) = \Tilde{x}^2 - n \equiv \Tilde{x}^2 \pmod{n}$ ดังนั้นจะสรุปได้ว่า
\begin{equation}
    Q(x_{i_1})Q(x_{i_2}) \ldots Q(x_{i_r}) \equiv (\Tilde{x}_{i_1}\Tilde{x}_{i_2}\ldots\Tilde{x}_{i_r})^2 \pmod{n}
\end{equation}
ให้ $\Tilde{x}_{i_1}\Tilde{x}_{i_2}\ldots\Tilde{x}_{i_r} = b $ จะได้ว่า $b^2 \equiv a^2 \pmod{n},$ $a^2 - b^2 \equiv 0 \pmod{n}$ ทำให้ $n|(a-b)$ และ $n|(a+b)$ ซึ่งอาจจะได้ตัวประกอบใหม่ของ $n$ (ที่ไม่ใช่ $1$ หรือ $n$) \cite{landquist_2001}
\subsection{การกำหนดฐานตัวประกอบ และช่วงของตะแกรงที่ใช้}

เมื่อได้ตั้งหลักการและเงื่อนไขของจำนวนที่จะใช้ในการแยกตัวประกอบแล้ว เราจะต้องการวิธีที่มีประสิทธิภาพในการหา $x_i$ และการเลือกผลคูณของ $Q(x_i)$ ที่เป็นจำนวนกำลังสอง ถ้าเกิดสังเกตดูแล้ว การที่จำนวนหนึ่งจะเป็นจำนวนกำลังสอง จำนวนนั้นจะต้องมีตัวประกอบเฉพาะที่มีเลขชี้กำลังเป็นคู่ทุกตัว ดังนั้น เราจำเป็นจะต้องหาตัวประกอบเฉพาะ และเลขชี้กำลังของแต่ละตัวของ $Q(x)$ ทุกตัว (รวมถึง $-1$ ด้วยถ้าหาก $Q(x) < 0$) เนื่องจากจำนวนอาจมีขนาดมาก และตัวประกอบอาจมีความซับซ้อน ทำให้เราควรจำกัดค่า $Q(x)$ ให้มีค่าน้อยๆ และจำกัดให้ตัวประกอบอยู่ภายในเซตของจำนวนเฉพาะค่าน้อย (รวม $-1$) ซึ่งเราจะเรียกเซตนี้ว่าฐานตัวประกอบ (Factor Base)

\vspace{4mm}

ในการจะจำกัดให้ค่า $Q(x)$ มีค่าน้อยๆ เราจะจำกัด $x$ ให้มีค่าใกล้ $0$ กล่าวคือสำหรับจำนวน $M$ ที่จะนิยามเป็นขอบเขตของตะแกรง (Bound) และกำหนดให้ $x \in [-M,M]$ เป็นช่วงของตะแกรง (Sieve Interval) แล้วทำการคำนวณในช่วงที่กำหนดขึ้นเพื่อประหยัดระยะเวลาที่ใช้ในการทำงาน (บางตำราอาจกำหนดนิยาม $Q(x) = x^2-n$ และให้ $x \in [\lfloor\sqrt{n}\rfloor-M,\lfloor\sqrt{n}\rfloor+M]$ ซึ่งสามารถใช้ได้เหมือนกัน แต่การคำนวณแบบนี้จะเห็นภาพได้ชัดเจนกว่า)

\vspace{4mm}

ในการเลือกฐานตัวประกอบนั้น ลองสังเกตว่า ถ้าให้ $x$ อยู่ในช่วงตะแกรง และให้จำนวนเฉพาะ $p$ หาร $Q(x)$ ลงตัว นั่นก็คือ $(x-\lfloor\sqrt{n}\rfloor)^2 \equiv n \pmod{p}$
จะทำให้ $n$ เป็นส่วนตกค้างกำลังสองในมอดุโล $p$ สรุปได้ว่า จำนวนเฉพาะคี่ที่อยู่ในฐานตัวประกอบทุกตัวจะต้องเป็นไปตามคุณสมบัติของ Legendre Symbol คือ
\begin{equation}
    \Big(\frac{n}{p}\Big) = 1 \equiv n^{\big(\dfrac{p-1}{2}\big)} \pmod{p}
\end{equation}
เงื่อนไขอีกอย่างหนึ่งก็คือ เราจะเลือกจำนวนเฉพาะมาเป็นฐานตัวประกอบ โดยเน้นจำนวนเฉพาะที่มีค่าน้อยๆ โดยมีค่าน้อยกว่าขอบเขต $B$ ที่กำหนด ซึ่งค่าของ $B$ ที่เหมาะสมจะขึ้นอยู่กับขนาดของ $n$ โดยปกติทั่วไป การหาจำนวนฐานตัวประกอบ $n(B)$ ที่เหมาะสมมักจะใช้สูตรข้างต้น (จากบทความของ Pomerance \cite{canfield_erdös_pomerance_1983})
\begin{equation}
    n(B) = e^{\big(\sqrt{\ln(n)\ln(\ln(n))}\big)^{\sqrt{2}/4}}
\end{equation}
เมื่อพิจารณาแล้ว พบว่าขอบเขตของตะแกรงมีค่าเป็นกำลังสามของค่านี้ คือ
\begin{equation}
    M = e^{\big(\sqrt{\ln(n)\ln(\ln(n))}\big)^{3\sqrt{2}/4}}
\end{equation} 

เมื่อทราบค่า $B$ จะทำให้สามารถนิยามค่าของ $Q(x)$ ได้ว่า $Q(x)$ จะต้องเป็นจำนวนที่มีคุณสมบัติ $B$-$smooth$ กล่าวคือ ตัวประกอบเฉพาะที่มากที่สุดของ $Q(x)$ ต้องมีค่าไม่เกิน $B$ นั่นเอง

\subsection{การกรองค่าในตะแกรงกำลังสอง}
หลังจากการกำหนดฐานตัวประกอบและช่วงของตะแกรงเรียบร้อยแล้ว เราจะทำการเลือก $x$ จากช่วงของตะแกรง ทำการคำนวณหา $Q(x)$ และแยกตัวประกอบเพื่อตรวจสอบว่าเป็นจำนวน $B$-$smooth$ หรือไม่ แล้วเก็บไว้เฉพาะจำนวนที่เป็น $B$-$smooth$
การคำนวนเหล่านี้อาจทำได้ง่ายในฐานตัวประกอบที่มีจำนวนน้อย แต่ขาดประสิทธิภาพอย่างรวดเร็ว หากขนาดและจำนวนของตัวประกอบมีค่าสูงขึ้น
ดังนั้น เราจะใช้หลักการของตะแกรงเพื่อทำการแยกตัวประกอบจำนวนเหล่านี้พร้อมกัน (ถ้าหากเราทำงานในระบบกระจาย เราสามารถแบ่งการทำงานเป็นส่วน ๆ ตามช่วงของตะแกรงได้)

\vspace{4mm}

ถ้าเราให้ $p$ เป็นตัวประกอบเฉพาะของ $Q(x)$ เราสามารถพิสูจน์ได้ว่า $p \mid Q(x+p)$ และ $Q(x) \equiv Q(y) \pmod{p}$ เมื่อ $x \equiv y \pmod{p}$
ดังนั้น สำหรับ $p$ ดังกล่าว เราสามารถเปลี่ยนการแยกตัวประกอบเป็นการหาคำตอบของสมการ
\begin{equation}
    Q(x) = (x+\lfloor\sqrt{n}\rfloor)^2 - n = s^2 \equiv 0 \pmod{p}, \quad x \in \mathbb{Z}_p
\end{equation}
การหาคำตอบของสมการนี้สามารถทำได้จากขั้นตอนวิธีของ Tonelli--Shanks ซึ่งก็คือการหารากที่สองของ $Q(X)$ มอดุโล $p$ หลังจากการใช้ขั้นตอนวิธีดังกล่าว
จะได้คำตอบของสมการ $2$ คำตอบคือ $s_{p_1}$ และ $s_{p_2} = p - s_{p_1}$ ซึ่งจะทำให้ $p \mid Q(x_i)$ เมื่อ $x_i = s_{p_1} + pk$ หรือ $x_i = s_{p_2} + pk$ และ $k \in \mathbb{Z}$

\vspace{4mm}

เมื่อเรารู้ค่า $s_{p_1}$ และ $s_{p_2}$ เรียบร้อยแล้ว เราสามารถนำค่านี้ไปผ่านขั้นตอนตะแกรงได้หลากหลายวิธี ทดลองสังเกตตัวอย่างการทดลองคำนวณ $s_{p_1}$ และ $s_{p_2}$ ประกอบกับการทดลองหารตามปกติในตารางด้านล่าง

\begin{center}
\begin{tabular}{ |c|cccccccccc|c| } 
 \hline
 $x$ & $1$ & $2$ & $3$ & $4$ & $5$ & $6$ & $7$ & $8$ & $9$ & $10$ & \\
 $\Tilde{x}$ & $86$ & $87$ & $88$ & $89$ & $90$ & $91$ & $92$ & $93$ & $94$ & $95$ & \\ 
 \hline
 $Q(x)$ & $9$ & $182$ & $357$ & $534$ & $713$ & $894$ & $1077$ & $1262$ & $1449$ & $1638$ & $Power: 0$\\
 \hline
 $p: 2$ & & \checkmark & & \checkmark & & \checkmark & & \checkmark & & \checkmark & $s_2 = \{0\}$\\
 $p: 3$ & \checkmark & & \checkmark & \checkmark & & \checkmark & \checkmark & & \checkmark & \checkmark & $s_3 = \{0,1\}$ \\
 $p: 7$ & & \checkmark & \checkmark & & & & & & \checkmark & \checkmark & $s_7 = \{2,3\}$  \\
 $p: 13$ & & \checkmark & & & & & & & & \checkmark & $s_{13} = \{2,10\}$ \\
 $p: 17$ & & & \checkmark & & & & & & & & $s_{17} = \{3,14\}$  \\
 \hline
 $\to$ & $3$ & $1$ & $1$ & $89$ & $713$ & $149$ & $359$ & $631$ & $69$ & $3$ & $Power: 1$ \\
 \hline
 $p: 2^2 = 4$ & & & & & & & & & & & $s_4 = \varnothing$ \\
 $p: 3^2 = 9$ & \checkmark & & & & & & & & & \checkmark & $s_9 = \{0,1\}$ \\
 $\vdots$ & & & & & & & & & & & $\vdots$ \\
 \hline
 $\to$ & $1$ & $1$ & $1$ & $89$ & $713$ & $149$ & $359$ & $631$ & $69$ & $1$ & $Power: 2$\\
 \hline
\end{tabular}
\end{center}

จะพบว่า เมื่อเรารู้เซต $s_p$ ของค่า $p$ แต่ละตัวแล้ว สามารถนำไปตัด --- หรือในที่นี้คือ หาร $p$ ออกจาก $Q(x_i)$ ได้เลย ถ้า $x_i\equiv s_{p_k} \pmod{p}, \ \  s_{p_k} \in s_p$  เหมือนกับเป็นตะแกรงของเอราทอสเทนีสที่ตัดทุกจำนวนที่มี $p$ เป็นตัวประกอบ แต่ในกรณีนี้ เราจะตรวจสอบแค่ในส่วนต่างกำลังสอง นักคณิตศาสตร์จึงตั้งชื่อขั้นตอนวิธีนี้ว่า "ตะแกรงกำลังสอง" นั่นเอง

\vspace{4mm}

ถ้าหากอธิบายอย่างละเอียดก็คือ ในแต่ละรอบ เราจะทำการหาร $p$ แต่ละตัวในฐานตัวประกอบทีละตัวจนครบ เก็บข้อมูลจำนวนครั้งที่หารไปแล้ว $\mod 2$ ไปเรื่อยๆ แล้วตรวจสอบค่าทั้งหมดว่ามีค่าไหนที่เป็น $1$ (หารลงตัวทั้งหมด) นำค่านั้นมาเก็บไว้จนกว่าจะมีจำนวนเพียงพอ ซึ่งควรใช้อย่างน้อย $n(B) + 1$ ตัว หรือเท่ากับจำนวนในฐานตัวประกอบ บวก $1$ ถ้าหากยังไม่พอ ให้กลับไปวนรอบใหม่ แต่เพิ่มค่ามอดุโลจาก $p$ เป็น $p^2$ ไปเรื่อยๆ

อันที่จริงแล้ว ยังมีวิธีที่จะลดเวลาการทำงานได้อีกหลายวิธี เช่น การกำหนดเงื่อนไขเพื่อทดลองนำค่าที่ตัวประกอบเหลือน้อยๆ มาแยกตัวประกอบ (Threshold Value) หรือการขยายค่าของ $Q(x)$ ให้รองรับพหุนามกำลังสอง เพื่อที่จะลดปริมาณการคำนวณลง (MPQS) หรือวิธีอื่นๆ แต่ในบทความนี้จะมอบให้ผู้อ่านศึกษาเพิ่มเติม

\subsection{การหาผลคูณที่เหมาะสมด้วยการกำจัดแบบเกาส์ (Gaussian Elimination)}

ถ้า $Q(x)$ สามารถแยกตัวประกอบออกมาได้หมด เราจะทำการนำเลขชี้กำลัง (จำนวนครั้งที่หารได้) ในมอดุโล $2$ รวมกันเป็นแถวในแมทริกซ์ $A$ ยกตัวอย่างเช่น ถ้า $Q(x) = 1638 = 2 \cdot 3^2 \cdot 7 \cdot 13$ และฐานตัวประกอบเป็นเซต $\{-1,2,3,13,17,19,29\}$ เราจะได้แถวที่สอดคล้องกับ $Q(x)$ นี้คือ $(0,1,0,1,1,0)$ เมื่อเรานำค่าต่างๆ แต่ละค่าเก็บในแมทริกซ์จนเพียงพอ เราจะมีข้อมูลเพื่อหาตัวประกอบของ $n$
\vspace{4mm}
เป้าหมายของเราคือ เลือก $x_i$ มาอย่างน้อย $1$ จำนวน เพื่อที่จะให้ผลคูณ $Q(x_i)$ เป็นจำนวนกำลังสอง ซึ่งก็คือ เราต้องการให้ผลรวมของเลขชี้กำลังของตัวประกอบเฉพาะทุกตัวเป็นเลขคู่ หรือก็คือสมมูลกับ $0\pmod{2}$ สังเกตว่า เป็นไปได้ที่จะหาผลคูณ $Q(x_i)$ ได้หลายวิธี ถือว่าเป็นเรื่องที่ดี เพราะผลคูณอย่างน้อยครึ่งหนึ่งไม่ได้ช่วยให้หาตัวประกอบของ $n$ ได้
ถ้าหากเรานิยามเป็นสมการ กล่าวคือ เมื่อให้ $Q(x_1), Q(x_2), \dots, Q(x_k)$ เราต้องการหาค่าของ
\begin{equation*}
    \sum_{i=1}^{k} Q(x_i)e_i
\end{equation*}
สำหรับ $e_i \in \{0,1\}$ ดังนั้น หากเรากำหนดให้ $\vec{a_i}$ แทนแถวที่เก็บเลขชี้กำลังมอดุโล $2$ ของ $Q(x_i)$ แสดงว่า เราต้องการ $e_i$ ที่
\begin{equation*}
    \sum_{i=1}^{k} \vec{a_i}e_i \equiv 0 \pmod{2}
\end{equation*}
ซึ่งก็คือ
\begin{equation*}
    \vec{e}A \equiv \vec{0} \pmod{2}
\end{equation*}
เมื่อ $\vec{e} = (e_1,e_2,\dots,e_k)$ ดังนั้น โจทย์ปัจจุบันจึงกลายเป็นการหาปริภูมิผลเฉลยของการคูณเวกเตอร์และแมทริกซ์ ซึ่งสามารถทำได้โดยขั้นตอนวิธีการตัดออกของกัสเซียน หรือการดำเนินการแบบแมทริกซ์อื่น ๆ เช่น การใช้แมทริกซ์สลับเปลี่ยนในการเปลี่ยนรูปสมการเป็น $A^T\vec{e}\equiv\vec{0}\pmod{2}$ เพื่อให้คำนวณได้ง่ายขึ้น
\vspace{4mm}
เมื่อทำการตัดออกแล้ว จะได้ $\vec{e}$ ที่เป็นปริภูมิผลเฉลยออกมา เราก็ทำการหา $a =\sqrt{\prod_{i=0}^{k} Q(x_i)}$ และ $b = \prod_{i=1}^{k} \Tilde{x}$ แล้วนำสองค่านี้มาเปรียบเทียบกับ $n$ คือหา $r = (b-a,n)$ ถ้า $r \not\in \{1,n\}$ จะได้ $r$ และ $\frac{n}{r}$ เป็นตัวประกอบของ $n$ เราสามารถนำสองค่านี้ไปตรวจสอบจำนวนเฉพาะ แล้วอาจจะแยกตัวประกอบต่อให้ครบถ้วนถ้าต้องการ 
%--------------------------------------
%--------------------------------------
\section{ตัวอย่างการแยกตัวประกอบด้วยขั้นตอนวิธีตะแกรงกำลังสอง}
สมมติว่าเราต้องการแยกตัวประกอบ $N = 7387$ พิจารณา $p$ ที่ค่าต่าง ๆ เพื่อหาฐานตัวประกอบที่เหมาะสม

\begin{center}
\begin{tabular}{ |c|cccccccccc| } 
 \hline
 $p$ & $2$ & $3$ & $5$ & $7$ & $11$ & $13$ & $17$ & $19$ & $23$ & $29$ \\ 
 $\big( \frac{n}{p} \big)$ & $\varnothing$ & $1$ & $-1$ & $1$ & $-1$ & $1$ & $1$ & $-1$ & $1$ & $-1$ \\ 
 \hline
\end{tabular}
\end{center}

จำนวนฐานตัวประกอบที่เหมาะสม ควรมีค่าประมาณ $n(B) = \big(e^{\sqrt{\ln{(7387)}\ln{(\ln{(7387)})}}}\big)^{\sqrt{2}/4} \approx 5.42$ ดังนั้นเราจะใช้จำนวนเฉพาะ $5$ จำนวนแรกคือ $\{2,3,7,13,17\}$ รวมกับ $-1$ เป็นฐานการแยกตัวประกอบ

เพื่อความสะดวก เราจะพิจารณา $x \geq \lfloor\sqrt{7387}\rfloor$ นั่นคือ $x \geq 85$ แล้วทำการหา $Q(x)$ ที่สามารถแยกตัวประกอบได้ทั้งหมดในฐานที่กำหนดไว้ หลังจากการหา $\Tilde{x}$ อย่างน้อย $n(B)+1 = 6$ จำนวน ได้ค่า $Q(x)$ ที่เหมาะสมดังนี้

\begin{center}
\begin{tabular}{ |c|c|c|cccccc|c| } 
 \hline
 \multicolumn{3}{|c|}{จำนวนที่เลือก} & \multicolumn{7}{|c|}{จำนวนตัวประกอบ $\pmod{2}$} \\
 \hline
 $x$ & $\Tilde{x}$ & $Q(x)$ & $-1$ & $2$ & $3$ & $7$ & $13$ & $17$ & แจกแจง \\ 
 \hline
 $1$ & $86$ & $9$ & $0$ & $0$ & $0$ & $0$ & $0$ & $0$ & $9 = 3^2$ \\ 
 $2$ & $87$ & $182$ & $0$ & $1$ & $0$ & $1$ & $1$ & $0$ & $182 = 2\cdot 7 \cdot 13$ \\ 
 $3$ & $88$ & $357$ & $0$ & $0$ & $1$ & $1$ & $0$ & $1$ & $351 = 3\cdot 7 \cdot 17$ \\ 
 $10$ & $95$ & $1638$ & $0$ & $1$ & $0$ & $1$ & $1$ & $0$ & $1638 = 2 \cdot 3^2\cdot 7 \cdot 13$ \\ 
 $31$ & $116$ & $6069$ & $0$ & $0$ & $1$ & $1$ & $0$ & $0$ & $6069 = 3 \cdot 7\cdot 17^2$ \\
 $37$ & $122$ & $7497$ & $0$ & $0$ & $0$ & $0$ & $0$ & $1$ & $7497 = 3^2\cdot 7^2 \cdot 17$ \\
 \hline
\end{tabular}
\end{center}


พิจารณาค่าตัวประกอบใน $Q(x)$ ที่เลือก เมื่อแปลงเป็น Matrix แล้ว เราต้องการหา $\vec{e}$ ที่  $\vec{e}A = \vec{0}\pmod{2}$ หรือก็คือ

\begin{equation*}
\begin{bmatrix}
0 & 0 & 0 & 0 & 0 & 0 \\
0 & 1 & 0 & 1 & 0 & 0 \\
0 & 0 & 1 & 0 & 1 & 0 \\
0 & 1 & 1 & 1 & 1 & 0 \\
0 & 1 & 0 & 1 & 0 & 0 \\
0 & 0 & 1 & 0 & 0 & 1 
\end{bmatrix}
\cdot \vec{e} \equiv \vec{0} \pmod{2}
\end{equation*}
ซึ่งจะทำให้ได้คำตอบหนึ่งคือ $\vec{e} = (0,1,0,1,0,0)$ โดยจากคำตอบนี้สามารถสรุปได้ว่า
\begin{align*}
    b & = & 87 \cdot 95 = 8265 \equiv 878 &\pmod{7387}\\
    a & = & \sqrt{Q(2)\cdot Q(10)} = \sqrt{2^{(1+1)} \cdot 3^2 \cdot 7^{(1+1)} \cdot 13^{(1+1)}} = 2\cdot 3 \cdot 7 \cdot 13 \equiv 546 &\pmod{7387}
\end{align*}
ดังนั้น จะทำให้ $(b-a,n) = (332,7387) = 83$ และ $(b+a,n) = (1424,7387) = 89$ เป็นตัวประกอบของ $7387$

ในกรณีนี้มีอีกคำตอบหนึ่งคือ $\vec{e} = (0,0,1,0,1,1)$ ซึ่งสามารถสรุปได้ว่า
\begin{align*}
    b & = & 88 \cdot 116 \cdot 122 = 1245376 \equiv 4360 &\pmod{7387}\\
    a & = & \sqrt{Q(3)\cdot Q(31)\cdot Q(37)} = \sqrt{3^{(1+1+2)} \cdot 7^{(1+1+2)} \cdot 17^{(1+2+1)}} = 3^2 \cdot 7^2 \cdot 17^2 = 127449 \equiv 1870 &\pmod{7387}
\end{align*}

ซึ่งจะทำให้ $(b-a,n) = (2490,7387) = 83$ และ $(b+a,n) = (6230,7387) = 89$ เป็นตัวประกอบของ $7387$
\section{โปรแกรมสาธิต}
ผู้เขียนได้จัดทำโปรแกรมสาธิต เพื่อให้สามารถลองนำไปใช้ทดสอบการแยกตัวประกอบจริงได้ และให้สามารถทำความเข้าใจได้โดยไม่ยุ่งยากซับซ้อน โดยได้เขียนเป็นภาษา Java ได้แนวคิดโปรแกรมส่วนใหญ่จากโปรแกรมของผู้ใช้ Github \verb |Maosef| \cite{maosef} แต่โปรแกรมสาธิตดังกล่าวมีข้อเสียคือ การเก็บค่าในหน่วย \verb |long| ที่ทำให้แยกตัวประกอบด้วยค่าได้ไม่เกิน $2^{64}$ หรือประมาณ $9.223\cdot10^{18}$ ซึ่งมีค่าน้อยกว่าช่วงที่โปรแกรมจะทำงานได้มีประสิทธิภาพดีกว่าขั้นตอนวิธีอื่น รวมถึงการที่เขียนด้วยขั้นตอนวิธีในรูปแบบที่พื้นฐานที่สุด ทำให้ทำงานช้ากว่าโปรแกรมในปัจจุบันอย่างมาก \\

\vspace{2mm}

วิธีใช้: แก้ไขค่า \verb |7387| ในเมธอด \verb |qsieve.QuadraticSieve.main()| เป็นค่าอื่นที่ต้องการ หรือเรียกใช้เมธอด \\ \verb |qsieve.QuadraticSieve::printFactors(long)| จะได้รับคำตอบในรูปของ  \verb |String| หรือเรียกใช้ \\ \verb |qsieve.QuadraticSieve::quadraticSieve(long)| ให้ได้คำตอบเป็น \verb |long[]| \\

\vspace{2mm}

โปรแกรมสาธิตถูกบันทึกไว้ที่ \url{https://github.com/pongtaewin/QuadraticSievePaper}
\renewcommand\refname{\vskip -1cm}
\section{เอกสารอ้างอิง}
\bibliographystyle{unsrtnat}
\bibliography{References}


\definecolor{codegreen}{rgb}{0,0.6,0}
\definecolor{codegray}{rgb}{0.5,0.5,0.5}
\definecolor{codepurple}{rgb}{0.58,0,0.82}
\definecolor{backcolour}{rgb}{0.95,0.95,0.92}

\lstdefinestyle{mystyle}{
    backgroundcolor=\color{backcolour},   
    commentstyle=\color{codegreen},
    keywordstyle=\color{magenta},
    numberstyle=\tiny\color{codegray},
    stringstyle=\color{codepurple},
    basicstyle=\ttfamily \scriptsize,
    breakatwhitespace=false,         
    breaklines=true,                 
    captionpos=b,                    
    keepspaces=true,                 
    numbers=left,                    
    numbersep=5pt,                  
    showspaces=false,                
    showstringspaces=false,
    showtabs=false,                  
    tabsize=2
}

\lstset{style=mystyle}


\section{ภาคผนวก: โปรแกรมสาธิต}
ไฟล์ที่ 1 : \verb|QuadraticSieve.java|
\lstinputlisting[language=Java]{code/QuadraticSieve.java}

ไฟล์ที่ 2 : \verb|SmoothFinder.java|
\lstinputlisting[language=Java]{code/SmoothFinder.java}

\newpage

ไฟล์ที่ 3 : \verb|LinearAlgebra.java|
\lstinputlisting[language=Java]{code/LinearAlgebra.java}

\newpage
ไฟล์ที่ 4 : \verb|ModOperators.java|
\lstinputlisting[language=Java]{code/ModOperators.java}

ไฟล์ที่ 5 : \verb|Pair.java|
\lstinputlisting[language=Java]{code/Pair.java}

ไฟล์ที่ 6 : \verb|Utility.java|
\lstinputlisting[language=Java]{code/Utility.java}

ไฟล์ที่ 7 : \verb|ValueNotFoundException.java|
\lstinputlisting[language=Java]{code/ValueNotFoundException.java}

\end{document}